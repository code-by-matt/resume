\documentclass[10pt]{article}

% helpful packages
\usepackage{geometry}
\usepackage{indentfirst}
\usepackage{enumitem}
\usepackage{hyperref}

% global formatting tweaks
\pagestyle{empty}
\geometry{margin=0.5in}
\setlength\parindent{9pt}
\setlist{nosep}
\renewcommand\labelitemi{--}

% new commands!
\newcommand{\name}[1]{\begin{center}\section*{\huge #1}\end{center}}
\newcommand{\topinfo}[1]{\begin{center}\vspace{-0.2cm}#1\vspace{-0.2cm}\end{center}}
\newcommand{\resumesection}[1]{\vspace{-0.2cm}\section*{#1}\vspace{-0.2cm}\hrule\vspace{0.2cm}}



\begin{document}

\name{Matthew Yuan}
\topinfo{\href{https://github.com/code-by-matt}{github.com/code-by-matt}}
\topinfo{\href{mailto:my4@princeton.edu}{my4@princeton.edu}}
\topinfo{609-216-0038}

\resumesection{Education}

\textbf{Princeton University}, Princeton, NJ \hfill September 2017--May 2021 (Expected)
\begin{itemize}
	\item Bachelor of Arts (A.B.) in Mathematics, 3.8 GPA.
	\item Coursework includes Algorithms and Data Structures, Probability and Stochastic Systems, Computational Geometry, Linear Algebra, Multivariable Calculus, Combinatorics, Real Analysis, and Fourier Analysis.
\end{itemize}

\textbf{University of Oxford}, Oxford, UK (study abroad) \hfill January 2020--Present
\begin{itemize}
	\item One on-site trimester at Worcester College, one remote trimester.
	\item Coursework includes Artificial Intelligence (AI), Algebraic Number Theory, Topology, Philosophy of Mathematics.
\end{itemize}

\resumesection{Skills}

\textbf{Programming Languages and Frameworks}
\begin{itemize}
	\item Java, Python, HTML/CSS, JavaScript, Node.js, Express, Socket.io, Cypress, Django, PostgreSQL, Bootstrap, Git, Heroku, \LaTeX.
\end{itemize}

\textbf{Concepts}
\begin{itemize}
	\item Object-Oriented Programming, Machine Learning, Mathematical Reasoning, Teaching.
\end{itemize}

\textbf{Languages}
\begin{itemize}
	\item English, Mandarin.
\end{itemize}

\resumesection{Experience}

\textbf{Adversarial Machine Learning Researcher}, University of Oxford \hfill March 2020--Present
\begin{itemize}
	\item Study and implement existing attacks against deterministic neural networks.
	\item Propose and test new attacks against Bayesian neural networks, which are known to be more robust.
	\item Meet weekly (remotely) with a research assistant and a doctoral student to discuss progress.
\end{itemize}

\textbf{Course Assistant}, Princeton University \hfill September 2018--May 2019
\begin{itemize}
	\item Led weekly problem sessions for about 50 students in Real Analysis and Linear Algebra.
	\item Helped students understand complex mathematical ideas and guided students through homework problems.
\end{itemize}

\resumesection{Projects}

\textbf{Thue-Morse Connect Four}, \href{https://tmc4.herokuapp.com}{tmc4.herokuapp.com}  \hfill October 2018--August 2019
\begin{itemize}
	\item Built a variant of Connect Four to explore what happens when players take turns following the Thue-Morse sequence. Intended to eliminate the game's first-player advantage. Two users can play each other on two different devices in real-time.
	\item Started work in Python/Django, then switched to Node.js/Express with Socket.io, using Cypress for testing.
\end{itemize}

\textbf{Seam Carving}, class project \hfill April 2019
\begin{itemize}
	\item Implemented an image resizing algorithm in Java that preserves an image's content without cropping or stretching.
	\item Achieved by using Dijkstra's algorithm to find minimal-energy seams in an image.
\end{itemize}

\resumesection{Activities}

\textbf{Writer}, \href{https://medium.com/@my4}{medium.com/@my4} \hfill November 2018--Present
\begin{itemize}
	\item Write short stories on Medium about math, origami, and whatever else interests me in the moment.
	\item Articles featured by Medium on their Math and Design topic pages and published by Medium's largest active publication.
	\item Previously managed and wrote for a student-run publication that interviews startup founders.
\end{itemize}

\textbf{Author in Princeton Undergraduate Research Journal} \hfill Spring 2019
\begin{itemize}
	\item Presented a creative, narrative explanation of Carl Friedrich Gauss's discovery that the regular seventeen-sided polygon is constructible using a compass and straightedge.
	\item Driven by curiosity and a desire to understand the real-world historical context of Gauss's work.
	\item 1 of 5 papers selected for publication out of 23 total submissions.
\end{itemize}

\end{document}
