\documentclass[10pt]{article}

% helpful packages
\usepackage{geometry}
\usepackage{indentfirst}
\usepackage{enumitem}
\usepackage{hyperref}

% global formatting tweaks
\pagestyle{empty}
\geometry{margin=0.5in}
\setlength\parindent{9pt}
\setlist{nosep}
\renewcommand\labelitemi{--}

% new commands!
\newcommand{\name}[1]{\begin{center}\section*{\huge #1}\end{center}}
\newcommand{\topinfo}[1]{\begin{center}\vspace{-0.2cm}#1\vspace{-0.2cm}\end{center}}
\newcommand{\resumesection}[1]{\vspace{-0.2cm}\section*{#1}\vspace{-0.2cm}\hrule\vspace{0.2cm}}



\begin{document}

\name{Matthew Yuan}
\topinfo{\href{https://github.com/code-by-matt}{github.com/code-by-matt}}
\topinfo{\href{mailto:my4@princeton.edu}{my4@princeton.edu}}
\topinfo{609-216-0038}

\resumesection{Education}

\textbf{Princeton University}, Princeton, NJ \hfill September 2017--May 2021 (Expected)
\begin{itemize}
	\item Bachelor of Arts (A.B.) in Mathematics, 3.8 GPA.
	\item Coursework includes Algorithms and Data Structures, Probability and Stochastic Systems, Computational Geometry, Linear Algebra, Multivariable Calculus, Combinatorics, Real Analysis, and Fourier Analysis.
\end{itemize}

\textbf{Machine Learning} by Stanford University, \href{https://www.coursera.org/learn/machine-learning}{coursera.org/learn/machine-learning} \hfill May 2019--August 2019
\begin{itemize}
	\item Studied supervised and unsupervised learning algorithms: linear regression, logistic regression, neural networks, support vector machines (SVMs), k-means clustering, principal component analysis (PCA), anomaly detection.
	\item Studied tools to evaluate and debug machine learning systems: bias/variance, learning curves, error analysis, ceiling analysis.
\end{itemize}

\resumesection{Skills}

\textbf{Programming Languages and Frameworks}
\begin{itemize}
	\item Java, Python, HTML/CSS, JavaScript, Node.js, Express, Socket.io, Cypress, Django, PostgreSQL, Bootstrap, Git, Heroku, \LaTeX.
\end{itemize}

\textbf{Concepts}
\begin{itemize}
	\item Object-Oriented Programming, Machine Learning, Mathematical Reasoning, Teaching.
\end{itemize}

\textbf{Languages}
\begin{itemize}
	\item English, Mandarin.
\end{itemize}

\resumesection{Experience}

\textbf{Algebraic Geometry Research} under Professor J\'{a}nos Koll\'{a}r, Princeton University \hfill June 2019--August 2019
\begin{itemize}
	\item Studied the connection between ideals of polynomial rings and algebraic varieties in affine and projective space.
	\item Met weekly with Prof. Koll\'{a}r as part of a 7-person research group.
\end{itemize}

\textbf{Course Assistant}, Princeton University \hfill September 2018--May 2019
\begin{itemize}
	\item Led weekly problem sessions for about 50 students in Real Analysis and Linear Algebra.
	\item Helped students understand complex mathematical ideas and guided students through homework problems.
\end{itemize}

\resumesection{Projects}

\textbf{Thue-Morse Connect Four}, \href{https://tmc4.herokuapp.com}{tmc4.herokuapp.com}  \hfill October 2018--August 2019
\begin{itemize}
	\item Built a variant of Connect Four to explore what happens when players take turns following the Thue-Morse sequence. Intended to eliminate the game's first-player advantage. Two users can play each other on two different devices in real-time.
	\item Started work in Python/Django, then switched to Node.js/Express with Socket.io, using Cypress for testing.
\end{itemize}

\textbf{Seam Carving}, class project \hfill April 2019
\begin{itemize}
	\item Implemented an image resizing algorithm in Java that preserves an image's content without cropping or stretching.
	\item Achieved by using Dijkstra's algorithm to find minimal-energy seams in an image.
\end{itemize}

\resumesection{Activities}

\textbf{Author in Princeton Undergraduate Research Journal}, \href{https://bit.ly/2W72vBR}{bit.ly/2W72vBR} \hfill Spring 2019
\begin{itemize}
	\item Presented a creative, narrative explanation of Carl Friedrich Gauss's discovery that the regular seventeen-sided polygon is constructible using a compass and straightedge.
	\item Driven by curiosity and a desire to understand the real-world historical context of Gauss's work.
	\item 1 of 5 papers selected for publication out of 23 total submissions.
\end{itemize}

\textbf{Editor of Profiles in Entrepreneurship}, \href{https://medium.com/profiles-in-entrepreneurship}{medium.com/profiles-in-entrepreneurship} \hfill October 2018--Present
\begin{itemize}
	\item Manage a team of 4 writers for an intercollegiate publication that provides student entrepreneurs actionable advice from startup founders and VCs.
	\item Produced over 30 articles in the 2018--2019 school year.
\end{itemize}

\end{document}
